\section{Advanced Git}
\headlineframe{Advanced Git}

\begin{frame}[c, fragile]{Partial Adding}
  \begin{itemize}
    \item Commits should be small, logically contained units
    \item Sometimes, we implement multiple things in one go 
    \item Go through all changes interactively, select what we want to add with
      \begin{code}{bash}
        $ git add -p
      \end{code}
  \end{itemize}
\end{frame}

\headlineframe{%
  Changing the git history\\
  (aka the danger zone)
}%

\begin{frame}[c]{Disclaimers}
  \begin{itemize}
    \item The main/main branch's history should only be modified under severe circumstances\\
      \begin{itemize}
        \item Sensitive data in the history\footnote{Under most circumstances I wouldn't recomend to change the history. Change the passwords.}
        \item Large files in the history that need to be removed
      \end{itemize}
    \item Not-yet-pushed commits can be freely modified
    \item Feature branches can usually be modified
    \item Most large projects will even ask you to cleanup the history of your Pull Request to have a \enquote{nice} history
    \item Modifying already-pushed commits requires pushing with the \mintinline{bash}+--force+ option
    \item The main/main branch should be protected against force pushes\\
      (Github/Gitlab settings)
  \end{itemize}
\end{frame}

\begin{frame}[c, fragile]{Fixing the last commit}
  \begin{itemize}
    \item Just changing the last commit is one of the most common use cases
      \begin{itemize}
        \item Fix a typo in the commit message
        \item Add a forgotten file
        \item Remove an accidentally commit file
      \end{itemize}
    \item Make and add the changes you want to include / fix in the last commit
    \item Execute
      \begin{code}{bash}
        $ git commit --amend
      \end{code}
      \begin{itemize}
        \item Adds the current staging area to the last commit (optional)
        \item Opens the editor for editing the commit message
        \item Overwrites the last commit (will change the hash)
      \end{itemize}
  \end{itemize}
\end{frame}

\begin{frame}[c]{Rebase - rewriting the git history}
  Rebase is a very powerful tool to rewrite the git history.

  It can
  \begin{itemize}
    \item Change commit order
    \item Drop / edit single commits
    \item Merge multiple commits into one
  \end{itemize}
\end{frame}


\begin{frame}[t, fragile]{Merge vs.\ Rebase}
  \begin{itemize}
    \item Default behaviour of \mintinline{bash}+git pull+ is equivalent to \mintinline{bash}+git fetch && git merge <remote>/<branch>+
    \item This results in a non-linear history with many merge conflicts like \enquote{Merging remote tracking branch...}
    \item \mintinline{bash}+git pull --rebase+ is equivalent to \mintinline{bash}+git fetch && git rebase <remote>/<branch>+
    \item It makes the history equal to the remote history, and then tries to apply the local commits in order
  \end{itemize}

  Can be made the default with \mintinline{bash}+git config --global pull.rebase true+

  Changes how conflicts are resolved: Instead of creating a single merge commit that 
  contains the fixes to make the two parents compatible, each commit that is rebased is adapted
  so the conflicts never existed.
\end{frame}

\begin{frame}[t, fragile]{\mintinline{bash}+git pull --rebase+ vs.\ \mintinline{bash}+git pull --merge+}
  \begin{columns}[onlytextwidth, t]
    \begin{column}{0.22\textwidth}
      Situation at start

      \begin{tikzpicture}
        \graph [
          layered layout,
          nodes={
            blue!70!black,
            node font=\ttfamily,
          },
        ]{
          a <- b <- c <- d <- e <- main [vertexDarkRed];
          b <- 1 <- 2 <- you [vertexDarkRed];
        };
      \end{tikzpicture}
    \end{column}
    \hfill
    \begin{column}{0.22\textwidth}
      With merging

      \begin{tikzpicture}
        \graph [
          layered layout,
          nodes={
            blue!70!black,
            node font=\ttfamily,
          },
        ]{
          a <- b <- c <- d <- e <- f;
          b <- 1 <- 2 <- 3;
          d <- 3;
          3 <- f;
          f <- main [vertexDarkRed];
        };
      \end{tikzpicture}
    \end{column}
    \hfill
    \begin{column}{0.22\textwidth}
      After rebase

      \begin{tikzpicture}
        \graph [
          layered layout,
          nodes={
            blue!70!black,
            node font=\ttfamily,
          },
        ]{
          a <- b <- c <- d <- e <-  main [vertexDarkRed];
          e <- 1 <- 2 <- you [vertexDarkRed];
        };
      \end{tikzpicture}
    \end{column}
    \hfill
    \begin{column}{0.22\textwidth}
      After merging the rebased branch

      \begin{tikzpicture}
        \graph [
          layered layout,
          nodes={
            blue!70!black,
            node font=\ttfamily,
          },
        ]{
          c <- d <- e <- f <- main [vertexDarkRed];
          e <- 1 <- 2;
          2 <- f;
        };
      \end{tikzpicture}
    \end{column}
  \end{columns}
\end{frame}


\begin{frame}[c, fragile]{Interactive Rebase}
  \begin{itemize}
    \item Very powerful tool to change commits 
    \item Joining / dropping / reordering / changing commits
  \end{itemize}
  \begin{code}{bash}
    $ git rebase -i <target commit>
  \end{code}
\end{frame}

\begin{frame}[c, fragile]{Submodules}
  \begin{itemize}
    \item Git discourages mono-repositories with many projects or
      just adding other projects to a repository
    \item Useful for 
      \begin{itemize}
        \item external source dependencies (e.\,g.\ Google Test for C++ projects)
        \item meta-repositories joining multiple repositores at specific versions
      \end{itemize}
    \item Submodules add a reference to another repository at a certain commit:
      \begin{code}{bash}
        $ git submodule add <url> <path>
      \end{code}
    \item Cloning does not include submodules by default, needs
      \begin{code}{bash}
        $ git clone <url> --recursive
      \end{code}
    \item Update submodules (e.\,g.\ if changed on the remote)
      \begin{code}{bash}
        $ git submodule update --init --recursive
      \end{code}
  \end{itemize}
\end{frame}
